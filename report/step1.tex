\section{Step 1 - Movement}
%Introduction
For the first step, we need to implement collision avoidance, such that two cars never inhabit the same space. This makes the alley a potential deadlock, since opposite cars would not be able to pass each other, so we also need to solve this. We first look at the collision problem before approaching the alley problem.

We must solve these problems using only semaphores. Monitors are not allowed in this step.

\subsection{Collision}
\subsubsection{Analysis}
Moving 8 (and 1 isolated, the toddler) cars around in their own threads, there is bound to be some conflicts. We have to avoid cars moving into other cars' spaces, but also avoid two cars moving into the same space at the same time. This forces us to use some kind of mutual exclusion, where only one car at a time can inhabit any space.

It is to our benefit, if we can assume that every iteration of the \texttt{run} loop always results in the car moving, i.e. it does not ``give up'' and try again in the next iteration. This will ease alley and barrier detection in future steps, and also avoid semi-busy waits when cars attempt to move into spaces in use. Therefore, we want cars to suspend execution when they are met with a non-empty space, until the space becomes empty. This can be done using some initially empty semaphore, in which the blocking car adds a token when it moves away.

If two cars attempt to move into the same space, only one must succeed. This can be solved by a semaphore initially with a single token, which allows a car to move into the space. Since only one of the cars will receive the semaphore token, only one will hold precedence and can move into the space. At some point, a token must be returned to this semaphore. If the first car has not moved yet from the contested position, the second car has a problem identical to the first case: attempting to move into inhabited space.

In addition to these problems, step 5 will need to remove cars for service, and the implementation will depend on how collision is handled. Also, since cars sleep during their movement, mutual exclusion is more difficult to implement since all cars move in two parts: moving half-way and completing the movement.

So in conclusion, solving these problems would require some semaphore use to suspend cars and/or mutual exclusion from moving.

\subsubsection{Design}
We solved this by creating a grid of semaphores the same size as the playground, so each position corresponds to a semaphore. All these semaphores contain a single token initially, which is removed when a car inhabits the space. When the car moves, it initially holds tokens from both its new position and its old one, only returning the old one when completing the move.

\subsubsection{Implementation}
\texttt{SemFields} is the class containing the grid of semaphores. It contains \texttt{P(Pos p)} and \texttt{V(Pos p)} methods, either of them performing the respective operation on the semaphore in the given position. It is not robust, and does not detect positions out of bounds.

In the \texttt{run} method of the \texttt{car} class, the original movement code has been modified to support this new paradigm. The car first requests the token from the semaphore in the \texttt{newpos} position. When it recieves this, the car holds tokens to both its current and future position, and continues normally with its animation and sleep. After this, the car changes curpos to newpos, and returns a token to its old position.

When cars are instantiated or retrieved from maintenance, they remove the token from their position.

\subsubsection{Discussion}
The \texttt{SemFields} are expensive in memory, but almost certainly the fastest solution, since all cars can move completely independently until imminent collision. We did however explore other solutions cheaper in memory.

Instead of a grid of semaphores, we tried a grid of booleans where each position represented if a car was using this space. We used a semaphore to make it mutually exclusive. The problem was that cars do not move instantaneously, but instead move halfway and sleep before completing the move. If the same car retained permission to the grid, all cars would be forced to wait, making it quite slow.

Another problem was handling cars waiting for each other. Since no semaphores were involved, we had to implement some so each car could wait for each other. In the end, we did develop a fairly complex system where cars would leave a semaphore in a separate grid when they waited. When a car finally completed its move, it would add a token to any semaphore which was attached to its position. This proved difficult however, when multiple cars was contested for the same spot and removal, especially when both happened at the same time. We would need lists of semaphores in each position, and easy removal of elements in these to solve both of those. In the end, it would probably not be cheaper than our current solution, and was already much more complex.

Since we never found a better solution, we settled for the current one.

Technically, we would only need semaphores on all spaces which at least two cars traverse, since these are the only spaces prone to collision.

\subsection{Alley}
\subsubsection{Analysis}
The alley is only one space wide. This means only one lane of cars with the same direction can move through the alley at any time. Since we are not allowed to change the path of cars, we cannot force them to turn around in the event of a collision. This means we have to predict future collisions, and not allow cars with opposite directions to move in the alley. Equivalently, all cars in alley at any time must always have the same direction.

This information is a critical region. Only one car should be able to read the state of the alley at any time, otherwise two cars could enter at the same time, which could lead to conflict. For example, if the alley is empty, two cars of opposite direction could attempt to enter at the same time. By using a semaphore to ensure mutual exclusion, only one car can enter at a time, and reading the alley's state is guaranteed not to be outdated until leaving the critical region.

Like with collision, when a car wishes to enter the alley, it must either immediately enter or suspend execution until it can. Thus cars wishing to enter the alley must attempt to remove tokens from some semaphore, which allows them to move into the alley. However, we want cars of same direction to enter the alley right after each other, so cars should only wait on this access semaphore if the direction of the alley is opposite their own.

It is preferred not to rely on collision code to solve these problems, though we could expect that only one car can attempt to enter the alley from the top at any time, and only two cars can attempt to enter the alley from below. Cars 5, 6, 7 and 8 all move in a straight line before entering the alley, but 1, 2 and 3, 4 enter from different paths. If 1 and 3 attempts to enter the alley at the same time, both should be allowed at the same time. Otherwise, either may wait for the other to exit the alley.

\subsubsection{Design and implementation}
For our solution, we used four semaphores, all of them binary with initially one token. Two of these, \texttt{bottom} and \texttt{top} are used exclusively by cars 1-4 (cars going clockwise) and 5-8 (cars going counter-clockwise), respectively. Both of these semaphores ensure that only one car from each direction is competing for access to the alley at a time, while the rest are waiting for either of these semaphores.

A car requests access to read and write the state of the alley by the semaphore \texttt{carsRegion} which encapsulates the critical regions of the alley. There are two scenarios:

\textbf{Car enters the alley}: If the alley contains cars of the opposite direction, the car will release access to \texttt{carsRegion} and request a token from the semaphore \texttt{access} which controls any access to the alley, regardless of direction. When it receives this token, \emph{or if the alley's direction is the same as its own}, it moves on to actually enter the alley. It does this by requesting \texttt{carsRegion} access and then updates the current direction of alley, \texttt{alleyDir}, (since it might be empty), increments number of cars currently in the alley. It then releases \texttt{carsRegion} access and its \texttt{top} or \texttt{bottom}. It is now moving in the alley.

\textbf{Car leaves the alley}: The car requests access to \texttt{carsRegion}, and then decrements the number of cars in the alley. If there are no cars left, the alleys direction is set to ``none'' and a token is put into \texttt{access} to allow others to enter.

\subsubsection{Discussion}

In fact we do not need the \texttt{top} semaphore, since it is guaranteed that only one of cars 5-8 are attempting to enter the alley at any time. We chose to keep it however, as the solution is more generally correct and will work for any shape of track (as long as the alley is defined correctly). Further it is easier to translate the algorithm into promella to be verified via spin.

The solution is not fair and can potentially cause starvation of one of the groups. This is illustrated best by using the ``slow'' feature of the GUI. It could be solved, but it is a design decision how to handle it; fairness could be based on rounds per individual car or by the sum of rounds in each group.

There is an optimization missing regarding the entering mechanism of car 1 and 2. They could enter a few steps earlier than car 3 and 4 when waiting for the other group to pass by, because the group will always pass 1 and 2 first. This would require splitting the alley into two sections and connect them in the correct manner.

% Could have been reader/writer approach