\section{Step 4 - Monitors}
For this step, we were required to re-implement the alley from step 2 and the barrier from step 3 but using monitors. A monitor is any class whose methods are declared with the \texttt{synchronized} keyword, which are the processes of a monitor. Only one thread can access any of those methods at any time, guaranteeing mutual exclusion across all of them.

Threads are enqueued FIFO by the \texttt{wait} method. \texttt{notify} wakes the first thread, and \texttt{notifyAll} wakes all the threads. Using these methods, we can implement the queues in monitors.

Compared to the semaphore solution, the monitor solutions for both barrier and alley are much shorter, simpler and elegant. They are much preferred to their semaphore counterparts.

When we refer to a \emph{while-wait} loop, it is short hand for a while-loop whose body only contains a \texttt{wait} call with the specified condition.

\subsection{AlleyMonitor}
\subsubsection{Analysis}
Compared the semaphore version, we can shorten our code a lot since mutual exclusion is guaranteed. In addition, we have access to queues now, so we don't need to restrict the amount of cars contested for alley access at a time.

In this case, we need to ensure that all cars attempting to enter the alley, either enters a while-wait loop or successfully gains permission to the alley. When the last car leaves, we want to wake the those waiting for permission.

\subsubsection{Design}
The design comes almost directly from the analysis. When a thread enters, it checks if it the alley contains any cars of the opposite direction. If it does, the thread enters a while-wait loop, only leaving when this condition changes. When the thread passes the while loop, it changes the direction of the alley and terminates the method, now allowed to move into the alley.

\subsubsection{Implementation}
\texttt{AlleyMonitor} is the subclass of \texttt{Alley} implemented with monitors. The overridden methods \texttt{enter(int no)} and \texttt{leave(no)} are both \texttt{synchronized}, making them mutual exclusive.

When a thread calls \texttt{enter} with the first argument being the car number, it enters a while-wait loop until the direction in the alley is not opposite of the cars direction. In this case, the alley is either empty (direction equals zero) or only contains cars going in the same direction, making it safe to enter. The car increments the variable cars, assigns the direction of the alley equal to its own (which only has an effect in case the alley was empty) and terminates the method.

The leave method decrements the cars variable, possibly resulting in cars equal to zero. In this case, the thread calls \texttt{notifyAll}, waking all the cars in the queue (cars which could not enter yet), and the direction of the alley is set to zero.

\subsubsection{Discussion}
Writing the monitor version of the alley was straightforward. It did not seem necessary to consider other solutions, since this one is simple and cost effective.

\subsection{BarrierMonitor}
\subsubsection{Analysis}
The barrier is much easier to solve with monitors, since we have access to queues. Instead of fiddling with semaphores gaining tokens equal to the amount of waiting cars, we can instead put cars into a while-wait loop, and release when the threshold has been exceeded.

We will still have to keep track of which cars are allowed to leave though, since any thread can wait an arbitrary amount of time before leaving the barrier. We would like any car to be able to leave when the threshold has been exceeded, no matter how long the car has waited.

\subsubsection{Design}
When a car enters the barrier, it tells the barrier it has arrived noting its number, and enters a while-wait loop. When enough cars has entered, the last car notices that the threshold has been exceeded, and signals the barrier that all waiting cars can leave. The last car can directly leave, since it was never waiting, but all the waiting cars are woken from their waiting loop so they can exit. The barrier wipes the record of waiting cars, now ready to receive new ones.

This solution contains each cars information individually in the barrier, such that faster cars do not steal slower cars' permissions.

\subsubsection{Implementation}
The \texttt{BarrierMonitor} class, extended from \texttt{Barrier}, is the barrier implemented using monitors. When instantiated, the integers \texttt{cars} and \texttt{threshold} is assigned $0$ and $9$ respectively. A boolean \texttt{on} is set to \texttt{false}, since the barrier is off by default. Two boolean arrays \texttt{waiting} and \texttt{pass}, both of length $9$ (the amount of cars), is initialized with all values \texttt{false}. The monitor contains the three required methods, \texttt{sync(int no)}, \texttt{on()} and \texttt{off()}, overridden from the super class with an additional \texttt{synchronized} keyword to ensure mutual exclusion. The method \texttt{setThreshold(int k)} is also overridden as \texttt{synchronized}.

When a thread/car calls the \texttt{sync} method, the argument being equal to the car number, \texttt{cars} is incremented and checked if it is less than or equal to \texttt{threshold}.

If it is, the car sets \texttt{waiting[no]} to \texttt{true} and \texttt{pass[no]} to \texttt{false}. The first array keeps track of waiting cars and the second signals when they can leave. The thread then enters a while-wait loop, only exiting it if/when the barrier is turned off or if \texttt{pass[no]} is \texttt{true}.

In case \texttt{cars} is less than or equal to \texttt{threshold}, the thread calls \texttt{notifyAll}, waking all waiting cars, and then for all $i=0..9$ assigns \texttt{pass[i]} to \texttt{true} if \texttt{waiting[i]} is \texttt{ true}.

After either of these cases, the method terminates, and the thread can continue execution of \texttt{run}, allowing the car to move through the barrier.

Turning the barrier on or off through \texttt{on} and \texttt{off} flips the value of \texttt{on} respectively. In addition, \texttt{on} will set all values of \texttt{waiting} to \texttt{false} and all of \texttt{pass} to \texttt{ true}, wiping the memory of the barrier. \texttt{off} calls \texttt{notifyAll}, allowing all waiting cars to leave. Since \texttt{on} wipes the memory of the barrier, all cars are able to leave even if the barrier is flipped from \texttt{on} and \texttt{off} quickly.

\subsubsection{Discussion}
The biggest hurdle was somehow signaling cars to know if they could leave, no matter how slow any car might be. The goal was to allow a car to leave when the threshold had been met, even if other cars had already queued up again. With a variable threshold, this meant other cars could pass a lot of times before some slow car would leave the barrier. An early solution was an integer \texttt{turn} which would increment every time the threshold was exceeded. Every car in the queue would save the value of \texttt{turn} when they entered the queue, and could tell if the barrier had been opened since they entered the queue. The problem, though improbable, was that \texttt{turn} might be incremented through all integers until it hit the exact same value before a car woke up.

The final solution is always safe, albeit uses a bit more memory (negligible with only $9$ cars). \texttt{pass[i]} would only become \texttt{false} when car $i$ entered the queue. Therefore the car can wait an arbitrary amount of time, and still pass the barrier. Both of the solutions tackled spurious wakeup by saving the state of the barrier, and comparing when the thread wakes up.

\subsection{Extra E}
We completed the extra step E, which required the threshold implemented for the barrier. Note that we did not implement this feature with the semaphore version. We already explained the usage of \texttt{threshold}, which decides when to release the cars in \texttt{enter}. \texttt{setThreshold} changes the value of \texttt{threshold}, immediately releasing all cars if it is exceeded.

Thus the solution works, even if the threshold has been set lower than the current amount of waiting cars. It also avoids any possible conflict with car $0$ because the barrier has separate records for each car.

There has not been attempts at alternative solutions, since the current one is simple and works without errors.

\subsubsection{Implementation}
The \texttt{setThreshold(int k)} method takes an integer argument, which \texttt{threshold} is assigned to. Then if it is less than or equal to \texttt{cars}, it releases the cars in the same manner as in \texttt{enter}.